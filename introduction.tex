\subsection*{Syfte}
Syftet med laborationen är att stifta bekantskap med relationen mellan det utstrålade ljuset från exciterade ämnen och energinivån hos de exciterade atomerna.
\subsubsection*{Frågeställningar:}
\begin{itemize}
	\item Räkna ut ett värde på Rydbergs konstant $R_H$ för väte
    \item Räkna ut ett värde på Rydbergs konstant $R_{Hg}$ för kvicksilver
\end{itemize}

\subsection*{Metod}
För att få ljus av just ett ämnes spektrum användes dels en vätgaslampa, dels en kvicksilverlampa. Dessa riktades in ett spektroskop i vilket det infallande ljuset spreds av ett gitter varvid våglängden av det infallande ljuset kan avgöras genom hur mycket det bryts.

\subsection*{Teori}
\subsubsection*{Brytning i gitter:}
När ljus av våglängd $\lambda$ träffar ett gitter med spaltavstånd $d$ i infallsvinkel $\theta_i$ kommer sambandet 
\begin{equation}
	d\left(\sin\theta_o-\sin\theta_i\right)=k\lambda: \quad k\in\mathbb{Z}
    \label{eq_gitterbrytning}
\end{equation}
där $\theta_o$ är den utgående vinkeln.

\subsubsection*{Rydbergs formel:}
När en atom går från ett exciterat tillstånd till ett lägre avges en foton med energin $E_{foton}=E_f-E_i$ där $E_f$ är energin av det slutgiltiga tillståndet och $E_i$ det initiala tillståndet. Fotonen kommer ha en våglängd $\lambda$ enligt Rydbergs formel:
\begin{equation}
	\frac{1}{\lambda}=R\left(\frac{1}{n^2}-\frac{1}{m^2}\right)
    \label{eq_Rydbergs_formel}
\end{equation}
där $n$ är det huvudkvanttalet för det initiala tillståndet $m$ är huvudkvanttalet för det slutgiltiga tillståndet, och $R$ är Rydbergs konstant beroende på vilket material som avger fotonen.