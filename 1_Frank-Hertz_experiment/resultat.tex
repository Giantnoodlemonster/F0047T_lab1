Med ett svep på $50~\textrm{V}$ accelerationsspänning, $U_A$, gavs följande vy, \Cref{fig:ollle} och värdena i \Cref{tab:maxmin}.
\begin{figure}[h]
	\centering
	\includegraphics[width = .8\textwidth]{osc_light_low_lowUA_e.jpg}
	\caption{x-led: accelerationsspänning, $U_A$, [$5~\textrm{V/Major div.}$]\\
			y-led: ström, $I_A$, [arbiträr enhet]}
	\label{fig:ollle}
\end{figure}

\begin{minipage}{\linewidth}
\begin{table}[H]
\centering
	\begin{tabular}{llll}
 	\textbf{Maxima [div.]}& \textbf{Maxima [V]}&\textbf{Minima [div.]}& \textbf{Minima [V]}\\\hline
	$3.4$&$17$&$3.0$&15\\
	$4.2$&$21$&$3.8$&19\\
	$5.0$&$25$&$4.6$&23\\
	$5.8$&$29$&$5.4$&27\\
	$6.6$&$33$&$6.2$&31\\
	$7.4$&$37$&$7.0$&35\\
	$8.2$&$41$&$7.8$&39\\
	$8.8$&$44$&$8.4$&42\\
	$9.0$&$45$&$ - $&
 	\end{tabular}
\caption{Maxima och minima i [divisioner] (för jämförelse mot graf) och [V] för analys. Data utläst från graf \cref{fig:ollle}}
\label{tab:maxmin}
\end{table}
\end{minipage}
\vspace{.5cm}

Från \Cref{fig:ollle} kan det ses hur strömmen ökar exponentiellt med högre accelerationsspänning med periodiska dipp som motsvarar excitationsenergierna för kvicksilver, $4~\textrm{eV}$, enligt periodiciteten som ses i \cref{tab:maxmin}. Den exponentiella ökningen ses tydligt i början av \Cref{fig:dark_high} där vi mätte maximan och miniman för systemet.
% TODO: "där vi testade extremerna för systemet" låter för ospecifikt tycker jag. Läsaren kommer ju inte ha någon aning om vad vi menar. 
% Good ?

Från jämförelse mellan bilderna \cref{fig:dark_lowub,fig:dark_highub} kan vi se att en skillnad i spänningen över filamentet ``förskjuter'' området vi ser maximan och miniman. I \cref{fig:dark_highub}, där $U_b$ har ökats, syns de periodiska fallen i strömmen bara mot slutet av skalan, där accelerationsspänningen är stor nog för att driva elektronerna förbi potentialbarriären. 

\begin{figure}[h!]
	\centering
	\begin{subfigure}[c]{0.47\textwidth}
	\includegraphics[width=\textwidth]{osc_dark_low_lowUA_e.jpg}
	\caption{Utströmmen mot accelerationsspänningen då en låg backspänning lagts på. Dalarna motsvarar de kinetiska energier varvid elektronerna stoppas av och exiterar kvicksilveratomerna.}
	\label{fig:dark_lowub}
	\end{subfigure}
	~
	\begin{subfigure}[c]{0.47\textwidth}
	\includegraphics[width=\textwidth]{osc_dark_low_highUA_e.jpg}
	\caption{Efter backspänningen ökat krävs det mer accelerationsspänning för att nå anoden och registrera som ström.}
	\label{fig:dark_highub}
	\end{subfigure}
	%\vspace{.5cm}
	~
	\begin{subfigure}[c]{0.47\textwidth}
	\includegraphics[width=\textwidth]{osc_dark_high_e.jpg}
	\caption{När temperaturen, backspänningen och spänningen över filamentet var perfekt visades tydligt det exponentiella förhållandet mellan accelerationsspänning och ström, och Franck-Hertz röret började agera som ett lysrör, se \cref{fig:striplight}.}
	\label{fig:dark_high}
	\end{subfigure}
	\caption{Bilder fångade på oscilloskopet med släckt lampa för tydligare bild}\label{fig:hgstuff}
\end{figure}
\begin{figure}[h!]
	\centering
	\begin{subfigure}[c]{0.47\textwidth}
	\includegraphics[width=\textwidth]{tube_dark_mid.jpg}
	\caption{motsvarar grafen i \cref{fig:dark_lowub}}
	\label{fig:fhlow}
	\end{subfigure}
	~
	\begin{subfigure}[c]{0.49\textwidth}
	\includegraphics[width=\textwidth]{tube_dark_high.jpg}
	\caption{Motsvarar grafen i \cref{fig:dark_highub}}
	\label{fig:fhmid}
	\end{subfigure}
	\vspace{.5cm}
	
	\begin{subfigure}[c]{0.47\textwidth}
	\includegraphics[width=\textwidth]{striplight.jpg}
	\caption{Motsvarar grafen i \cref{fig:dark_high}}
	\label{fig:striplight}
	\end{subfigure}
	\caption{Bilder fångade på Franck-Hertz röret med släckt lampa för tydligare bild, dessa motsvarar respektive grafer fångade i \cref{fig:hgstuff}}\label{fig:hgstuff}
\end{figure}
