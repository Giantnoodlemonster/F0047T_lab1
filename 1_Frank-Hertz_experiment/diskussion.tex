Om systemet ses från ett klassiskt mekaniskt perspektiv så består det av en elektron, en partikel, som får en initiell rörelseenergi från accelerationsspänningen, som måste färdas genom ett kontinuerligt retarderande fält i form av kvicksilvergasen och sedan passera potentialsteget som kommer från backspäningen innan den kan bidra till den uppmätta strömmen. 

För en given filamentspänning, kammartemperatur och backspänning kommer partikelflödet öka proportionerligt mot accelerationsspänningen, så den mätta strömmen borde ses som en ramp på oscilloskopet. På grund av backspänningen borde det ses en gräns längs accelerationsspänningsaxeln under vilken ingen ström syns då partiklarna inte har nog energi för att passera barriären. Om filamentspänningen ökas frigörs ett större antal elektroner som kan svepas av accelerationsspänningen, så rampen börjar på samma ställe men blir brantare. Om nu kammartemperaturen ändras så ändras mängden förångat kvicksilver; högre temperatur ger mer kvicksilver som kan absorbera energi från elektronerna. Därför borde temperaturkontrollen ge liknande resultat som backspänningen; senare start på rampen då mer kvicksilver ger större total absorption efter elektronens färd genom molnet.

%Det som faktiskt observerats är BADASS RUBBER DUCKBADASS RUBBER DUCK.

Enligt tabellen \cref{tab:maxmin} har vi miniman i spänningar $15, 19, 23, 27, 31, 35, 39, 42~\textrm{V}$. Då effekten av backspänningen börjar kicka in på slutet observerar vi bara de första $6$ spänningarna. Det kan antas att med rätt justeringar på backspänningen bör detta mönster upprepas både ner till det första kvantstadiet och säkert även för högre spänningar, smartast hade nog varit att zooma in på det tidiga området på plats, men det blev inte av. Ett tydligt mönster på cirka $\Delta 4~\textrm{V/peak}$ kan urskiljas. Denna energi bör stämma överrens med en våglängd av ljus emmiterad av kvicksilver. Användandes \cref{eq:fin} i \cref{sec:met} ges en våglängd på cirka $302*10^{-9}~\textrm{m}$, eller $302~\textrm{nm}$, se \cref{eq:resultat}. Stoppas $E=4.9~\textrm{eV}$ in i ekvationen, vilket bör vara den korrekta spänningsskillnden, fås med ekvationen \cref{eq:fin} den förväntade våglängden $253~\textrm{nm}$\cite{fhbook}.
\begin{equation}\label{eq:resultat} \lambda = \sqrt{\frac{hc}{E}} = 302*10^{-9}~\textrm{m}\end{equation}

$302~\textrm{nm}$ är ett ljus gränsfall till ultraviolett, och bör därför inte gå att se med bara ögon. Då skenet i tuben, \cref{fig:hgstuff}, har ett tydligt turkost sken estimerat till någonstans i det blå spektrat mellan $450 - 495~\textrm{nm}$ vilket bör överrensstämma med en spänninsgskillnad på $\approx 2.7-3.1~\textrm{eV}$ enligt \cref{eq:fin}, är vi inte riktigt säkra på vad som hänt.

Möjliga felkällor är, i ordning av sannolikhet.
\begin{itemize}
	\item Fel antagande i härledning av ekvationen \cref{eq:fin}, någonting relevant kan ha missats. 
	\item Fel estimering av förstärkningssteget. Då förstärkning ansågs vara irrelevant sattes denna i botten vilket antogs vara förstärkning $\textrm{Gain} = 1$.
	\item Fel avläsning från oscilloskop.
	\item Dålig kombination av test-setup-parametrar
	\item Materiell avvikande från specifikation
\end{itemize}

%3. Using quantum mechanics how would you explain the connection between IC and UA.
%Give an account for the values of accelerating voltages for the different maxima and minima, present your results in a table. From these values deduce the excitation energy.
%Take great care in your explanation of the experiment. To facilitate your process of understanding you can perform an ’gedanken experiment’ like following an electron as it is moving from the source to the collector. Ask yourself questions like, where in the tube will excitations occur, what happens to the electron after exciting a mercury atom, if an electron of constant kinetic energy (like 1eV) would move from the glowing filament to the collector how long would this take, how much is this in comparison to the time it takes to complete one cycle (at 50Hz).
