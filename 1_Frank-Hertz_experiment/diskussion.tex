Om systemet ses från ett klassiskt mekaniskt perspektiv så består det av en elektron, en partikel, som får en initiell rörelseenergi från accelerationsspänningen, som måste färdas genom ett kontinuerligt retarderande fält i form av kvicksilvergasen och sedan passera potentialsteget som kommer från backspäningen innan den kan bidra till den uppmätta strömmen. 

För en given filamentspänning, kammartemperatur och backspänning kommer partikelflödet öka proportionerligt mot accelerationsspänningen, så den mätta strömmen borde ses som en ramp på oscilloskopet. På grund av backspänningen borde det ses en gräns längs accelerationsspänningsaxeln under vilken ingen ström syns då partiklarna inte har nog energi för att passera barriären. Om filamentspänningen ökas frigörs ett större antal elektroner som kan svepas av accelerationsspänningen, så rampen börjar på samma ställe men blir brantare. Om nu kammartemperaturen ändras så ändras mängden förångat kvicksilver; högre temperatur ger mer kvicksilver som kan absorbera energi från elektronerna. Därför borde temperaturkontrollen ge samma resultat som backspänningen; senare start på rampen då mer kvicksilver ger större total absorption efter elektronens färd genom molnet. 

Det som faktiskt observerats är BADASS RUBBER DUCKBADASS RUBBER DUCK.
