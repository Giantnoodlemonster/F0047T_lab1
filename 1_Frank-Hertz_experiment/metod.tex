Alla tester kördes utan förstärkning och värden är sparade med $5~\textrm{V/div.}$
För att enkelt kunna läsa av maximan och miniman för accelerationsenergin, $U_A$, för att i sin tur kunna beräkna excitationsenergin, måste korrekt backspänning och en bra temperatur i ugnen uppnås. Med \Cref{eq:debrog}\cite[s. 303]{ph} kan man hitta våglängd från rörelseenergi, vilket kan användas för att identifiera excitationsspektrat, kvicksilvers unika fingeravtryck.
\begin{equation}\label{eq:debrog}
	\lambda = \frac{h}{\sqrt{2mE_k}}
\end{equation}
där $h$ är plancks konstant, $m$ är massan av atomen och $E_k$ är den kinetiska energin.

Temperaturen ska sättas till $180^o~\textrm{C}$, detta för att säkerhetsställa rätt tryck i tuben. För låg temperatur leder till att det kan bli för få kvicksilveratomer för att märka topparna, och för högt tryck leder till att medelavståndet mellan kollisioner sjunker och slumpmässig rörelse höjs\cite{rochhand}.

Backspänningen agerar som en potentialstegsbarriär, när denna är för hög ``absorberas'' alla låga accelerationsspänningar bort och ``subtraherar'' från de passerande elektronernas energi. Är denna för låg syns nästan bara brus.
En bra nivå på backspänningen sattes till cirka $9~\textrm{V}$.