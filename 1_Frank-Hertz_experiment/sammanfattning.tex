Frank-Hertz experiment var $1914$ det första elektriska mätningarna att tydligt visa kvantnaturen av atomer och ge inblick i deras kvantenerginivåer. Frank-Hertz experimentet vi utför accelererar elektroner genom ett moln av upphettad kvicksilvergas och mäter strömmen för olika spänningar och observerar det emmiterade ljuset från röret.

I mätningar kommer vi fram till en kvantspänningsskillnad på $\Delta U_A = 4.0~\textrm{eV}$ mellan miniman, se \cref{tab:maxmin}; vilket beräknas till en våglängd på $302~\textrm{nm}$, se \cref{eq:resultat}. Detta bör resultera i ett ljus, ej synligt med bara ögon, nästan ultraviolet ljus. I själva verket lös röret med ett ljus närmare turkos, vilket bör ligga någonstans i det blå spektrat mellan $450 - 495~\textrm{nm}$ vilket bör ge en spänning på $\approx 2.76~\textrm{eV}$.

Enligt fler källor, däribland en rapport från Manchester University\cite{bmfrankhertz} rapporterar kvantnivåer på $4.9~\textrm{V}$, vilket bör resultera i en rimligare och mer passande våglängd för det observerade skenet. $302~\textrm{nm}$ är väldigt nära den andra förväntade våglängden på $297~\textrm{nm}\cite{fhbook}$. 

Trots avvikelsen från de ``korrekta'' värdena har syftet med utförandet varit tydligt och förståelsen är densamma. Möjliga fel diskuteras i \cref{sec:disc}.