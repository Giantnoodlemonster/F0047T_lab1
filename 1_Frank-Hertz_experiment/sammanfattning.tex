Frank-Hertz experiment var $1914$ det första elektriska mätningarna att tydligt visa kvantnaturen av atomer och ge inblick i deras kvantenerginivåer. Frank-Hertz experimentet vi utför accelererar elektroner genom ett moln av upphettad kvicksilvergas och mäter strömmen för olika spänningar och observerar det emmiterade ljuset från röret.

I mätningar kommer vi fram till en kvantspänningsskillnad på $\Delta U_A = 4~textrm{V}$ mellan miniman, se \cref{tab:ollle}; vilket beräknas till en våglängd på $410~\textrm{nm}$, se \cref{eq:resultat}. Detta bör resultera i ett sken mellan lila och blå, men närmare lila. I själva verket lös röret med ett ljus närmare turkos, vilket bör ligga någonstans i det blå spektrat mellan $450 - 495~\textrm{nm}$.

Enligt fler källor, däribland en rapport från Manchester University\cite{bmfrankhertz} rapporterar kvantnivåer på $4.9~\textrm{V}$, vilket bör resultera i en rimligare och mer passande våglängd för det observerade skenet..

Trots avvikelsen från de korrekta värdena har syftet med utförandet varit tydligt och förståelsen är densamma. Möjliga fel kan innefatta fel avläsning från oscilloskopet, missförstånd i förstärkningsmodulen (då denna läts sitta på minima, vilket antogs vara $1x$ förstärkning), missförstånd i beräkningen av energin. Det finns också mindre sannorlika felkällor såsom åldern på mätverktygen eller fel inställning på backspänningen och temperaturen, men detta borde endast resultera i ett biaseringsfel.